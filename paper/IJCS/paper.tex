%%
%% This is file `sample-manuscript.tex',
%% generated with the docstrip utility.
%%
%% The original source files were:
%%
%% samples.dtx  (with options: `manuscript')
%% 
%% IMPORTANT NOTICE:
%% 
%% For the copyright see the source file.
%% 
%% Any modified versions of this file must be renamed
%% with new filenames distinct from sample-manuscript.tex.
%% 
%% For distribution of the original source see the terms
%% for copying and modification in the file samples.dtx.
%% 
%% This generated file may be distributed as long as the
%% original source files, as listed above, are part of the
%% same distribution. (The sources need not necessarily be
%% in the same archive or directory.)
%%
%% The first command in your LaTeX source must be the \documentclass command.
\documentclass[manuscript,screen,review]{acmart}

%%
%% \BibTeX command to typeset BibTeX logo in the docs
\AtBeginDocument{%
  \providecommand\BibTeX{{%
    \normalfont B\kern-0.5em{\scshape i\kern-0.25em b}\kern-0.8em\TeX}}}

%% Rights management information.  This information is sent to you
%% when you complete the rights form.  These commands have SAMPLE
%% values in them; it is your responsibility as an author to replace
%% the commands and values with those provided to you when you
%% complete the rights form.
\setcopyright{acmcopyright}
\copyrightyear{}
\acmYear{}
\acmDOI{}

%% These commands are for a PROCEEDINGS abstract or paper.
%%\acmConference[Woodstock '18]{Woodstock '18: ACM Symposium on Neural
%%  Gaze Detection}{June 03--05, 2018}{Woodstock, NY}
%%\acmBooktitle{Woodstock '18: ACM Symposium on Neural Gaze Detection,
%%  June 03--05, 2018, Woodstock, NY}
%%\acmPrice{15.00}
%%\acmISBN{978-1-4503-XXXX-X/18/06}


%%
%% Submission ID.
%% Use this when submitting an article to a sponsored event. You'll
%% receive a unique submission ID from the organizers
%% of the event, and this ID should be used as the parameter to this command.
%%\acmSubmissionID{123-A56-BU3}

%%
%% The majority of ACM publications use numbered citations and
%% references.  The command \citestyle{authoryear} switches to the
%% "author year" style.
%%
%% If you are preparing content for an event
%% sponsored by ACM SIGGRAPH, you must use the "author year" style of
%% citations and references.
%% Uncommenting
%% the next command will enable that style.
%%\citestyle{acmauthoryear}

%%
%% end of the preamble, start of the body of the document source.
\begin{document}

%%
%% The "title" command has an optional parameter,
%% allowing the author to define a "short title" to be used in page headers.
\title{Ontology-based approach to creating semantic wikis}

%%
%% The "author" command and its associated commands are used to define
%% the authors and their affiliations.
%% Of note is the shared affiliation of the first two authors, and the
%% "authornote" and "authornotemark" commands
%% used to denote shared contribution to the research.
\author{Ortiz Troncoso, Alvaro}
\email{alvaro.OrtizTroncoso@mfn.berlin}
\orcid{0000-0001-7620-1907}
\affiliation{%
  \institution{Museum f\"ur Naturkunde - Leibniz Institute for Research on Evolution and Biodiversity}
  \streetaddress{Invalidenstrasse 43}
  \city{Berlin}
  \country{Germany}
  \postcode{10115}
}

%%
%% By default, the full list of authors will be used in the page
%% headers. Often, this list is too long, and will overlap
%% other information printed in the page headers. This command allows
%% the author to define a more concise list
%% of authors' names for this purpose.
%% \renewcommand{\shortauthors}{Trovato and Tobin, et al.}

%%
%% The abstract is a short summary of the work to be presented in the
%% article.
\begin{abstract}
Maintaining a semantic wiki is challenging. Coping with increasingly complex wikis led to the development of a methodical approach for simplifying the creation and maintenance of semantic wikis.
The methodical approach used involves modeling the semantic relationships underlying the information in the wiki as an ontology, and then programmatically creating the wiki from the ontology constructs.
A methodical approach for creating and maintaining wikis greatly simplifies the creation and maintenance of semantic wikis. Furthermore, reusing vocabularies and taxonomies throughout several projects becomes manageable.
The approach was implemented as open source software that maps ontology constructs to the wiki artifacts, implementation details are discussed.
Semantic wikis support research at the Museum of Natural History Berlin by providing services for developing and documenting collection management and imaging procedures or annotating archive material. Additionally to these internally available wikis, some wikis are available on the Web, as publication medium for conference presentations or providing additional material for exhibitions. A total of 16 semantic wikis are currently online on the museum's servers.
\end{abstract}

%%
%% The code below is generated by the tool at http://dl.acm.org/ccs.cfm.
%% Please copy and paste the code instead of the example below.
%%
\begin{CCSXML}
<ccs2012>
<concept>
<concept_id>10003120.10003130.10003233.10003301</concept_id>
<concept_desc>Human-centered computing~Wikis</concept_desc>
<concept_significance>500</concept_significance>
</concept>
</ccs2012>
\end{CCSXML}

\ccsdesc[500]{Human-centered computing~Wikis}

%%
%% Keywords. The author(s) should pick words that accurately describe
%% the work being presented. Separate the keywords with commas.
\keywords{semantic wikis, ontology engineering, collaborative authoring}


%%
%% This command processes the author and affiliation and title
%% information and builds the first part of the formatted document.
\maketitle

\section{Introduction}
Wikis are web platforms allowing for collaborative authoring of information. With certainty, the best known wiki is Wikipedia, but wikis are also being used to setup corporate intranets, private collaboration environments or knowledge bases. A semantic wiki is a wiki whose functionality has been extended to encode information in a structured and machine-readable way: information encoded semantically forms a "consistent logical web of data" \cite{berners1998}, that can be used for enriching information with metadata, checking the conformance of information against an agreed upon schema, or accessing and combining data from different sources \cite{oren2006}. Semantic wikis therefore combine the metadata and conformance aspects required by knowledge management, with the possibility to access and build-upon a corpus of collaboratively collected knowledge. 

Semantic wikis have been used at the Museum f\"ur Naturkunde Berlin (Museum of Natural History Berlin) since 2015. A total of 16 semantic wikis are currently online on the museum's server. Most of these wikis are part of the museum's intranet, and support research by providing services for developing and documenting collection management and imaging procedures or annotating archive material. On the other hand, some wikis are available on the Web, as publication medium for conference presentations or providing additional material for exhibitions. An example of a publicly accessible semantic wiki running at the Museum of Natural History Berlin is the Wiki supplementing the Panda exhibition (http://biowikifarm.net/v-mfn/panda\_en/index.php).

The construction and provisioning of wikis has been the subject of two research projects at the Museum \cite{patzschke2016}. The experience gained demonstrates that creating and maintaining a semantic wiki is challenging, a major difficulty being that the complexity of the information stored in the wiki tends to increase as the user base grows, and new ideas are integrated into the wiki's semantic structure \cite{kiniti2013}. This increase in complexity can take several forms: the number of document types increases, the number of attributes per document increases, external vocabularies are added to the wiki, category trees become more complex. On the other hand, complex information offers more possibilities for gaining new insights, as categories and properties can be combined through semantic queries into new data. For example, querying the biographies of collectors in the wiki of the museum's historical division can be visualized as a map showing where the museum has collected specimens in the past (see \cite{ortiz2016} for this and other examples of semantic queries).

\section{Methodical approach}
Coping with this increasing complexity led to the development of a methodical approach for simplifying the creation and maintenance of semantic wikis. One way to create a semantic wiki is to model the semantic relationships underlying the information in the wiki as an ontology, and then to customize a wiki to reflect the constructs described by the ontology \cite{diiorio2006}. 

At this point it might be necessary to clarify the concept of ontology. A basic definition of what an ontology is in the realm of knowledge engineering was given by Gruber and Borst (cited in \cite{corcho2003}): an ontology is made of clear-cut and formal definitions of the concepts in a domain, and these definitions are accepted by the community of practitioners of that domain. Another way to put Gruber and Borst's definition: an ontology is a collaboratively constructed, formalized mind map.
Another definition of ontology can be gained from information theory. \citeauthor{pieterse2014} differentiate between five types of fundamental information structures: linear lists, hierarchical taxonomies, poly-hierarchical lattices, thesauri combining hierarchies with various kinds of relations between concepts, and ontologies, which extend thesauri with a set of rules for inferring new knowledge. This definition emphasizes the role of logical inference rules, and paves the way for the role ontologies play in machine learning (see \citeauthor{zhou2007} for a road map).

Ontology engineering is a vast topic by itself, yet a simple methodology could follow these basic steps \cite{noy2001}: obtaining domain information by consulting experts and gathering specific vocabularies (e.g. in the case of Museum of Natural History Berlin, zoological taxonomy, stratigraphical vocabulary and others); subsequently, modeling the ontology in an external tool such as Prot\"eg\"e \cite{musen2015}, and storing the result as an RDF/XML file \cite{gandon2014}. Software implemented at the Museum automates the conversion of the ontology file into web pages and forms in the wiki. This software is distributed under an Open Source License (https://github.com/MfN-Berlin/RDF-to-SemanticWiki), in the hope that it can be useful for computer scientists researching computer-mediated collaboration and ontology engineering.

\section{Implementation details}
The process by which the conversion of the ontology into a wiki is automated is implemented as follows: the RDF/XML file representing the ontology is parsed into an in-memory semantic model. Then, for each object in the model, a data access object (DAO) is created using a factory class. A connection is established with the wiki's web service (API). Finally, each DAO object saves itself to the wiki, using the wiki's API. With an ontology as input, the software knows two commands: ``import'' creates the necessary wiki pages, ``delete'' removes the corresponding wiki pages. The layout of the wiki is customizable through templating: templates can be used to specify the order in which elements should appear, as well as to filter out or hide some elements. The software is written in Python, templates are in XSLT. The resulting wiki pages are compatible with the semantic extension to the popular MediaWiki software (https://www.semantic-mediawiki.org).

Ontology constructs are mapped to the wiki artifacts according to these rules: ontology classes are mapped to forms, templates and page categories; data attributes are mapped to property pages of the corresponding data type; relations between objects are rendered as links between pages. The cardinality of attributes is enforced through appropriate form input fields. Class inheritance is supported. Furthermore, navigation elements are added to the wiki, to enable creating, listing and editing content following the structure defined by the ontology. The information stored in the wiki using these forms and templates is thus structured and consistent, and can be accessed using semantic queries (https://www.semantic-mediawiki.org).

\begin{table*}
  \caption{Ontology to MediaWiki mapping}
  \label{tab1}
  \begin{tabular}{ccl}
    \toprule
     Ontology element & MediaWiki markup & Example \\
    \midrule
     Class & Template:Classname, Form:Classname, Category:Classname & The ontology class "Entry" is mapped to Template:Entry, Form:Entry and Category:Entry \\
     Datatype property & Property:propertyname, markup in the corresponding class' template & The property "HasDetails" of the class "Entry" is mapped to the page Property:HasDetails and a snippet of markup is added to the page Template:Entry \\
     Object property & The objects in the range are listed in the domain class' form & 
    \bottomrule
  \end{tabular}
\end{table*}

\section{Conclusions and future work}
The method presented here and the associated software support research at the Museum of Natural History Berlin by providing a way to create collaboration environments for working on complex scientific projects, while enforcing structure and consistency. Conceivably, the method and software could be applied to other research institutions and even to areas of industry that face similar knowledge management challenges. Furthermore, as ontologies can be reused and extended \cite{noy2001}, the methodical approach facilitates knowledge transfer between projects, as taxonomies and vocabularies found useful in one project can be used to construct a wiki for a second project. As discussed above, the implemented software supports the conversion of semantic classes, attributes and relations, class inheritance and cardinality. This is a subset of the constructs defined by the RDF/XML specification \cite{musen2015}. More work is needed to extend the possibilities of the software so as to pave the way for more expressive collaboration environments.

\section{Citations and Bibliographies}

\begin{verbatim}
  \bibliographystyle{ACM-Reference-Format}
  \bibliography{paper.bib}
\end{verbatim}

\section{Acknowledgments}
\begin{verbatim}
  \begin{acks}
Software development at Museum f\"ur Naturkunde Berlin was sustained in part by the project ``Knowledge Transfer Concept for Research Contents, Methods and Competences in Research Museums'', funded by the German Federal Ministry of Education and Research (BMBF), grant no. 01IO1632.
  \end{acks}
\end{verbatim}

\end{document}
\endinput
%%
%% End of file `sample-manuscript.tex'.
